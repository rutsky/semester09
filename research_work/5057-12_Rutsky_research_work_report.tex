% Report on research work for 10th semester in SPbSTU.
% Copyright (C) 2010-2011 Vladimir Rutsky <altsysrq@gmail.com>

\documentclass[a4paper,10pt]{article}

% Encoding support.
\usepackage{ucs}
\usepackage[utf8x]{inputenc}
\usepackage[T2A]{fontenc}
\usepackage[russian,english]{babel}

\usepackage{amsmath, amsthm, amssymb}

% Spaces after commas.
\frenchspacing
% Minimal carrying number of characters,
\righthyphenmin=2

% From K.V.Voroncov Latex in samples, 2005.
%\textheight=24cm   % text height
\textheight=25cm   % text height
\textwidth=16cm    % text width.
\oddsidemargin=0pt % left side indention
\topmargin=-1.5cm  % top side indention.
\parindent=24pt    % paragraph indent
\parskip=0pt       % distance between paragraphs.
\tolerance=2000
%\flushbottom       % page height aligning
%\hoffset=0cm
\pagestyle{empty}  % without numeration

% Indenting first paragraph.
\usepackage{indentfirst}

\usepackage{setspace}
%\linespread{1.5}

\usepackage{enumitem}
%\usepackage{datetime}
\usepackage{verbatim}

% from http://www.latex-community.org/forum/viewtopic.php?f=5&t=1383&start=0&st=0&sk=t&sd=a
\makeatletter
\renewcommand\paragraph{\@startsection{paragraph}{4}{\z@}%
  {-3.25ex\@plus -1ex \@minus -.2ex}%
  {1.5ex \@plus .2ex}%
  {\normalfont\normalsize\bfseries}}
\makeatother

\begin{document}
\selectlanguage{russian}

\begin{center}
\begin{spacing}{1.3}
  {\Large\bfseries Отчет по проделанной работе в рамках НИРС} \\
  {\large Руцкий Владимир Владимирович, гр.~5057/12, 10~семестр 2011~г.}
\end{spacing}
\end{center}

\noindent\textbf{Тема работы:}\quad Решение задачи построения ортофотоплана по аэрофотоснимкам с большим параллаксом \\
\textbf{Место выполнения:}\quad ЗАО <<Транзас Новые Технологии>> \\
\textbf{Руководитель:}\quad Ковалёв Антон Сергеевич, 
магистр прикладной математики и информатики, 
ведущий инженер-программист (руководитель группы) ЗАО <<Транзас Новые Технологии>>

\section{Постановка задачи}
Даны два аэрофотоснимка поверхности Земли, сделанных из камеры, установленной на низколетающем летательном аппарате
(высота съёмки над поверхностью Земли находится в диапазоне от 800 до 1600 метров).

Углы зрения камеры лежат в диапазоне от $30^{\circ}$ до $60^{\circ}$.
Направление съёмки отклоняется от вертикального вниз не более чем на несколько градусов.
Снимки получены с высоким пространственным разрешением от 15 до 40~см на точку.

Положения камеры в пространстве в момент съёмки известны с точностью до 50 метров. 
Снятые камерой области поверхности Земли, пересекаются не менее чем половиной своей площади,
и объекты, стоящие на снимаемой поверхности и снятые на обоих снимках, видны под разными углами.

На снимаемой поверхности Земли встречаются объекты, возвышающиеся над её поверхностью на высоту до 40 метров: 
строения, деревья; 
и объекты, опускающиеся на глубину до нескольких метров: канавы, рвы.
В результате съёмки таких объектов из разных точек возникает явление параллакса:
объекты, не лежащие на поверхности Земли, оказываются смещены относительно объектов, лежащих на поверхности Земли,
на разных снимках на разное расстояние.

Требуется по данным снимкам выбрать точки привязки и построить ортофотоплан.

\section{Элементы исследования и новизны}
Из-за параллакса значительно усложняется задача совмещения снимков. 
Например, при преобладании на снимках леса,
снимки могут быть ошибочно совмещены не по плоскости Земли, 
а по плоскости верхушек деревьев леса, и
тогда снятая на тех же снимках дорога может быть некорректно состыкована на результирующем изображении.
Особенно эта проблема проявляется на снимках городов.

Данная задача была изучена для съёмок с больших высот и более низкой детализацией снимков.

\section{Ожидаемый результат}
Решение поставленной задачи позволит улучшить качество ортофотопланов, 
получаемых из аэрофотоснимков, сделанных на низких высотах. 

\section{Среда разработки}

Платформа: Microsoft Windows XP.

Языки программирования: C++, Python.

Среда разработки: Microsoft Visual Studio, NetBeans.

\section{Выполненная работа}

% \subsection{Анализ востребованности решения поставленной задачи}
% Ортофотопланы являются неотъемлемой частью фотограмметрии.
% 
% Для получения ортофотоплана необходимо:
%    1) получить снимки Земли, 
%    2) сшить снимки.
% 
% Bсточники снимков: 
%    1) космическая съёмка, 
%    2) аэрофотосъемка.
%    
% Космическая съёмка: 
%    дорогая, 
%    не оперативная (требует погодных условий и наличия спутника), 
%    ограниченное пространственное разрешение, 
%    с большими ограничениями доступна для коммерческих организаций, 
%    но высокоточная.
% 
% Аэрофотосъемка: 
%    дешевая, 
%    оперативная, 
%    высокое пространственное разрешение, 
%    доступна для небольших организаций. 
% Высокоточные данные возможны только на дорогом и сложном в обращении 
%    оборудовании (\$380K).
% 
% Т.\,о. возникает задача: сшить снимки сделанные с летательного аппарата без 
% точной калибрации и без точной пространственной информации о положении 
% камеры.
% 
\subsection{Обзор литературы}
Поставленная задача относится к области знаний компьютерного зрения, 
основные методы совмещения снимков, соответствующих условиям задачи, описаны 
в~\cite{szelisky06alignstichtut, szelisky10compvis}:
\begin{enumerate}
  \item Прямое попиксельное совмещение~[?]~--- КРАТКОЕ ОПИСАНИЕ~--- ВЫВОД: вычислительно сложный подход. 
  При проявлении параллакса на значительной площади метод работает очень плохо.
  Точности позиций и ориентаций камер недостаточно для получения корректного результата.
  \item Восстановление плотного облака 3D точек~[?]~--- вычислительно сложный, но очень перспективный подход.
  \item Восстановление информации о глубине пикселей, рассматривая исходные изображения как стереопару~[?]~--- 
  точности исходных данных недостаточно для построения модели эпиполярной геометрии и последующей ректификации изображений.
  \item Совмещение по особенным точкам изображения (feature-based)~[?]~--- наиболее гибкий подход.
\end{enumerate}

%TODO: MPPS (tr-2004-48.pdf)
%NOTE: Методы убирающие параллакс сдвигом - сдвиг не в ту плоскость приведёт
%к построению ортофотоплана со смещённой системой координат относительно 
%реальной.
%NOTE: Снимки из космоса - параллакс незаметен.

%NOTE: Старая аэрофотосъёмка - производятся частые снимки небольшой области строго
%под летательным аппаратом - параллакс незаметен.

\subsection{Выбранный метод решения}

Для построения ортофотоплана был выбран метод, основывающийся на выборе и сопоставлении особенных точек изображения [?].
Общая схема алгоритма построения ортофотоплана состоит в следующем:
\begin{enumerate}
  \item На снимках выбираются особенные точки $\{ \mathbf{p}_i \}$.
  \item Каждой особенной точке $\mathbf{p}_i$ ставится в соответствие дескриптор $\mathbf{v}_i \in \mathbf{R}^n$, 
  обладающий следующим свойством: 
  фрагменты снимков, соответствующие особенным точкам $\mathbf{p}_1$, $\mathbf{p}_2$, тем больше похожи друг на друга, 
  чем меньше евклидово расстояние в $\mathbf{R}^n$ между их дескрипторами $d = |\mathbf{v}_1 - \mathbf{v}_2|$.
  \item Между снимками ищется множество пар близких друг к другу особенных точек $S = \{ (\mathbf{p}_i, \mathbf{p}_j ) \},$
  где $\mathbf{p}_i$ из первого снимка, а $\mathbf{p}_j$ из второго.
  \item По найденным парам соответствующих особенных точек $S$ производится оценка матрицы гомографии $\mathbf{H} \in \mathbf{R}_{3, 3}$, 
  задающей преобразование между точками снимков: $\mathbf{p}_1 = \mathbf{H} \cdot \mathbf{p}_2$, где $\mathbf{p}_1$ и $\mathbf{p}_2$ точки
  первого и второго снимков соответственно (в однородных координатах).
  Также на этом этапе из $S$ выбрасываются пары, соответствующие ошибочно совмещённым особенным точкам изображений.
\end{enumerate}

Bundle ajdustment c Алгоритмом Левенберга—Марквардта


%TODO: Feature-detection - инвариантность к масштабу в данной задаче не нужна.
%TODO: Статистический метод для feature совмещения: 
%большой разброс - деревья (т.к. тень).
%TODO: Совмещение по теням
%TODO: Восстановление облака точек.

\subsection{Возникшие проблемы}
\paragraph{Искажение изображения линзами}%
Абберации оптической системы фотокамеры вносили искажения в исходные снимки 
и приводили к систематическим ошибкам.
Абберации были аппроксимированы полиномиальной моделью радиального искажения, 
описанной в~\cite{forsythponce04compvis} и~\cite{szelisky10compvis}, 
и убраны обратными преобразованиями.

TODO: Выбор точек привязки (keypoint/feature) так, чтобы вошли точки с Земли.

\subsection{Результаты}


\subsection{Дальнейшие исследования}
TODO: Ориентирование на водную поверхность при сшивке.

TODO: Ускорение за счет переноса вычилений на GPU.

\section{Литература}

\begin{thebibliography}{9}

\bibitem{forsythponce04compvis}
  Д.\,Форсайт, Ж.\,Понс,
  \emph{Компьютерное зрение. Современный подход}.
  М.: Издательский дом {\guillemotleft}Вильямс{\guillemotright},
  2004.

\bibitem{szelisky06alignstichtut}
  R.\,Szeliski,
  \emph{Image Alignment and Stitching: A Tutorial}.
  Now Publishers Inc,
  2006.

\bibitem{szelisky10compvis}
  R.\,Szeliski,
  \emph{Computer Vision: Algorithms and Applications}.
  Springer,
  2010.

\end{thebibliography}

\begin{comment}
%\begin{flushbottom}
\begin{flushright}
\begin{spacing}{1.5}
\begin{tabular}{l l l}
Студент & \noindent\underline{\makebox[2.5cm][l]{}} & В.\,В.\,Руцкий \\
Руководитель & \noindent\underline{\makebox[2.5cm][l]{}} & А.\,С.\,Ковалёв\\
\end{tabular}
\vspace{1cm}

\today
\end{spacing}
\end{flushright}
%\end{flushbottom}
\end{comment}
\end{document}
